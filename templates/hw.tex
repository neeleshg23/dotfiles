\documentclass[12pt]{article}
\usepackage[utf8]{inputenc}
\usepackage{amsmath,amssymb,hyperref,array,xcolor,multicol,verbatim,mathpazo}
\usepackage[normalem]{ulem}
\usepackage[pdftex]{graphicx}

\newenvironment{problem}[2][Problem]{\begin{trivlist}
\item[\hskip \labelsep {\bfseries #1}\hskip \labelsep {\bfseries #2.}]}{\end{trivlist}}

\begin{document}

%%%% In most cases you won't need to edit anything above this line %%%%

\title{\vspace{-4cm}Math 395 Homework 1}
\author{Neel Gupta}
\maketitle

\begin{problem}{1}
Prove that $\sqrt{2}$ is irrational. 
\end{problem}


\textit{Answer:}

Assume that $\sqrt{2}$ is rational in the form $\frac{p}{q}$ where $p,q \in \mathbb{Z}$ and share no common factors.

\begin{align*}
	(\frac{p}{q})^2 &= (\sqrt{2})^2 \\
	\frac{p^2}{q^2} &= 2 \\
	p^2 &= 2q^2
\end{align*}

which implies that $p^2$ is even. If $p^2$ is even, then $p$ is even, which implies that $p^2$ is divisible by 4. Hence $q^2$ is even, so $q$ is even. Since $p, q$ are even, they share a common factor which is a contradiction. Thus, $\sqrt{2}$ is irrational.
$\square$

\begin{problem}{7}
Prove that $\forall n \in \mathbb{N}$ and $\forall x \in \mathbb{R},|\sin nx| \leq n|\sin x|$.
\end{problem}

\textit{Answer:}

Let $P(n): |\sin nx| \leq n|\sin x|$.

For $P(1): |\sin x| = |\sin x| \therefore P(1)$ is true.

Let $I_0H_0:$ Assume $P(k)$ for some $k\in\mathbb{N}$.

For $P(k+1):$

\begin{align*}
|\sin((k+1)x)| &= |\sin(kx+x)| \\
&= |\sin kx \cos x + \cos kx \sin x| \\
&\leq |\sin kx||\cos x|+|\cos kx||\sin x| \\
&\leq |k|\sin x||*|\cos x| + |\cos kx| |\sin x| && \text{(by $I_0H_0$)} \\
&\leq k |\sin x| + |\sin x| && (\because |\cos x| \leq 1 \text{ and } |\cos kx| \leq 1) \\ 
&= (k+1)|\sin x| \\
&\therefore P(k+1) \text{is true.}
\end{align*}

By mathematical induction, $P(n)$ is true $\forall n \in \mathbb{N}$ and $\forall x \in \mathbb{R}.$
$\square$

\begin{problem}{8}
Prove that $\forall n \in \mathbb{N}, n^n \leq (n!)^2$.
\end{problem}

\textit{Answer:}

$$
n^n = \prod_{i=1}^{n} n 
$$

Similarly, 
\begin{align*}
(n!)^2 &= (1*2*...*n)(1*2*...*n)\\
&=(1*1)(2*2)...(n*n)\\
&=(1*n)(2*(n-1))(3*(n-2))...((n-1)*2)(n*1)\\
&=\prod_{i=1}^{n} i(n+1-i) && (\because \text{ Gauss' Trick})
\end{align*}

$$
\prod_{i=1}^{n} n \leq \prod_{i=1}^{n} i(n+1-i)
$$

\begin{problem}{9}
Prove that $\forall n \in \mathbb{N}$ 
$$
1 + 2 + ... + n = \frac{n(n+1)}{2}.
$$
\end{problem}

\textit{Answer:}

Let $P(n): 1 + 2 + ... + n = \frac{n(n+1)}{2}$.

For $P(1): \frac{1(1+1)}{2} = 1 \therefore P(1)$ is true.

Let $I_0H_0:$ Assume $P(k)$ is true for some $k \in \mathbb{N}$.

For $P(k+1):$
\begin{align*}
1 + 2 + ... + k + (k+1) = \frac{k(k+1)}{2} + k + 1 && \text{(by $I_0H_0$)} \\
= \frac{k(k+1) + 2(k+1)}{2} \\
= \frac{k^2+k+2k+2}{2} \\
= \frac{k^2+3k+2}{2} \\
= \frac{(k+1)(k+2)}{2} \\
\therefore P(k+1) \text{ is true.}
\end{align*}

By mathematical induction, $P(n)$ is true $\forall n \in \mathbb{N}$.


\begin{problem}{10}
Prove that $\forall n \in \mathbb{N}$ 
$$
1^2 + 2^2 + ... + n^2 = \frac{n(n+1)(2n+1)}{6}.
$$
\end{problem}

\textit{Answer:}

Let $P(n): 1^2 + 2^2 + ... + n^2 = \frac{n(n+1)(2n+1)}{6}.$

For $P(1): \frac{1(2)(2+1)}{6} = 1 = 1^2 = 1 \therefore P(1)$ is true.

Let $I_0H_0:$ Assume $P(k)$ is true for some $k \in \mathbb{N}$.

For $P(k+1)$:
\begin{align*}
\\ 1^2 + 2^2 + ... + k^2 + (k+1)^2 = \frac{k(k+1)(2k+1)}{6} + (k+1)^2 &&\text{(by $I_0H_0$)}
\\ = \frac{k(k+1)(2k+1)+6(k+1)^2}{6} 
\\ = \frac{(k^2+k)(2k+1)+6(k^2+2k+1)}{6} 
\\ = \frac{2k^3+3k^2+k+6k^2+12k+6}{6}
\\ = \frac{2k^3+9k^2+13k+6}{6}
\end{align*}

$ \pm \frac{1}{2},1,\frac{3}{2}, 2,3,6$ are the possible zeros of $2k^3+9k^2+13k+6$ by Rational Root Theorem. 

\begin{align*}
\\ = \frac{(k+1)(2k^2+7k+6)}{6}
\\ = \frac{(k+1)(k+2)(2(k+1)+1)}{6}
\\ \therefore P(k+1) \text{is true.}
\end{align*}

By mathematical induction, $P(n)$ is true $\forall n \in \mathbb{N}$.



\end{document}
